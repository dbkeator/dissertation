\chapter{Introduction}

Artifical Intelligence is defined as the study and design of intelligent systems. An intelligent system is one which perceives its environment and takes actions that maximizes its chances of success. The nature of this environment varies from system to system. For example, the environment for a chess playing system can be enumerated with a set of rules; the environment for an autonomous car is the size, position of cars, their relative speeds on the street and the position of important objects like traffic lights, stop signs. The search space of an AI problem is the total number of candidate solutions. In both the above examples, the search space contains an enormous of candidates. But for the second case, the problem of enumerating the search space is non-trivial. It requires a large amount of knowledge about the environment. Also, the interactions between elements in the real world are very hard to predict. They need to be \textit{sensed} in a real time fashion for decisions to be made accordingly. With the growing amount of sensors in the real world, its becoming possible to record events and activities of increasingly finer granularities. Sensory inputs in the real world could range from news articles on the web to stock market tickers to thermometer readings from wildlife parks. Such a large amount of heterogenous information cannot be easily stored, processed and analyzed. 

This dissertation addresses the problem of constructing computational representations of real world enviornments from various heterogenous sensors.

\section{Approach}
Things get more interesting when we realize that for the car application above we realize that the information at any instant is the description of the environment only at that point in time, and around a small spatial area around the car. There is no need to represent the entire world to help navigate the car. In this dissertation, we utilize this observation to construct a model of the environment which is most relevant to the intelligent system.

This dissertation presents a technique to construct computational models of the environment from various data sources. Examples of data source include social media web services to provide information about events and entities like Facebook, Twitter; services which can be queried to find information about places like Yelp; Sensors on personal mobile phones, for examples GPS which inform where the person is present at any given point in time. Given a seed description of the environment, the technique incrementally discovers the other events, entities and their relations which constitues the current environment. We refer to such this technique as progressive discovery. 

Progressive discovery involves two main components. First, knowledge of the world, what the various types of information in the world, and how they are related. Second, it uses a description of data sources to decide which source can contribute additional knowledge to what is already known about the environment. By querying the sources, the new data obtained is used to grow the current model of the environment. 


\section{Overview}
This dissertation is organized into the following chapters. Chapter 2 provides an overview of context, and how context has been used to address problem in various communities. Chapter 3 describes the related work in computer science uptil now, and how this work is informed by them. Chapter 4 describes our context discovery framework, how it models various data sources, and how our progressive discovery algorithm constructs models for real world problems. We facilitate this discussion with an example real world application to tag faces of people in personal photos. Chapter 5 analyzes the algorithmic complexity of different parts of the system, and provides experiments to confirm the same. We also present experiments to confirm the efficacy of our approach in the light of the real world application. Chapters 6 and 7 describe two extensions to the CueNet framework to solve problems of missing context and that of source selection. Finally, chapter 8 attempts to describe the future possibilities of context discovery.

\section{Terminology}
