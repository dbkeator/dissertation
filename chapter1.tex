\chapter{Introduction}

Artifical Intelligence is defined as the study and design of intelligent systems. An intelligent system is one which perceives its environment and takes actions that maximizes its chances of success. The nature of this environment varies from system to system. For example, the environment for a chess playing system can be enumerated with a set of rules; the environment for an autonomous car is the size, position of cars, their relative speeds on the street and the position of important objects like traffic lights, stop signs. The search space of an AI problem is the total number of candidate objects over which a decision needs to made. In both the above examples, the search space contains an enormous of candidates. But for the second case, the problem of enumerating the search space is non-trivial. It requires a large amount of knowledge about the environment. Also, the interactions between elements in the real world are very hard to predict. They need to be \textit{sensed} in a real time fashion for decisions to be made accordingly. With the growing amount of sensors in the real world, its becoming possible to record events and activities of increasingly finer granularities. Sensory inputs in the real world could range from news articles on the web to stock market tickers to thermometer readings from wildlife parks. Such a large amount of heterogenous information cannot be easily stored, processed and analyzed. 

This dissertation addresses the problem of constructing computational representations of real world enviornments from various heterogenous sensors, to reason which parts of the search space can be pruned without hurting the overall performance of the intelligent system. We refer to such a representation as the \textbf{Context Network} of the environment. The network describes real-world events occuring in the environment, the entities participating in them, and their semantic inter-relationships.

\section{Approach}
Before embarking on a mission to model the entire world, we ask ourselves the question: How much of the real world information is actually relevant to the intelligent system? Constructing the entire world model is extremely challenging and often unnecessary. For example, there might not be much value in representing sports events in New York in a model being used to help cars navigate in Japan. 

This dissertation presents a \textbf{\textit{progressive discovery}} algorithm to ingest information from various real world data sources to construct context networks containing the most relevant information for pruning the search space for the system. Examples of data sources include social media web services to provide information about events and entities like Facebook, Twitter; services which can be queried to find information about places like Yelp; Sensors on personal mobile phones, for example GPS which inform applications of the location of a person is present at any given point in time.

\textbf{What is progressive discovery?} Progressive discovery is an incremental process  where knowledge of real world events and entities can be added to a given context network. If we look at this definiton recursively, it says that given a context network and some data sources describing events and entities, a progressive discovery algorithm will recursively obtain information from the sources and relate it to context network. By repeatedly executing this algorithm, we can grow a context network until the data sources can provide no further information or the information in the network prunes the search space well enough for the AI problem to be fully solved.

In the following chapters, context networks, and their discovery from various data sources will be done in conjunction with an application to \textbf{tag faces in personal photos}. The face tagging algorithm, whose search space contains a few billion entities is a very hard real world AI problem. But if a real world model of the world existed, the search space which is relevant to this photo contains just the entities who are present within the field of view of the camera at the time the photo was captured. 

\section{Overview}
This dissertation is organized into the following chapters. Chapter 2 provides an overview of context, and how context has been used to address problems in various communities. Chapter 3 describes the related work in computer science uptil now, and how this work is informed by them. Chapter 4 describes our context discovery framework, how it models various data sources, and how our progressive discovery algorithm constructs models for real world problems. We facilitate this discussion with an example real world application to tag faces of people in personal photos. Chapter 5 analyzes the algorithmic complexity of different parts of the system, and provides experiments to confirm the same. We also present experiments to confirm the efficacy of our approach in the light of the real world application. Chapters 6 and 7 describe two extensions to the CueNet framework to solve problems of missing context and that of source selection. Finally, chapter 8 attempts to describe the future possibilities of context discovery.

\section{Terminology}
Before starting the discussion on Context Network, it is necessary to include this short note on terminology. We use the word `Object' to collectively refer to events and entities. An entity includes persons, places in the world, for example `Starbucks, UC Irvine', `The Eiffel Tower, Paris, France', or organizations, for example `Google Inc', `Royal Society of London'. `Object' has been used in literature to refer to things which have no spatio-temporal properties. But, in our discussion, an `object' could imply an event which exhibits spatio-temporal properties.



%Constructing the entire world model is an extremely challenging problem. We also realize that for most real world applications, the most interesting information at any instant is the description of the environment only at that point in time, and around a relatively small spatial area. For example, there is no need to represent the entire world to help navigate the car. We will utilize this observation to construct a model of the environment which is most relevant to the intelligent system.
%
%This dissertation presents a novel technique to construct context networks from various data sources. Examples of data source include social media web services to provide information about events and entities like Facebook, Twitter; services which can be queried to find information about places like Yelp; Sensors on personal mobile phones, for examples GPS which inform where the person is present at any given point in time. Given a seed description of the environment, the technique incrementally discovers the other events, entities and their relations which constitues the current environment. We refer to such this technique as progressive discovery. 
%
%In the following chapters, context networks, and their discovery from various data sources will be done in conjunction with an application to tag faces in personal photos. The face tagging algorithm, whose search space contains a few billion entities is a very hard real world AI problem. But if a real world model of the world existed, the search space which is relevant to this photo contains just the entities who are present within the field of view of the camera at the time the photo was captured.
%
%Progressive discovery involves two main components. First, knowledge of the world, what the various types of information in the world, and how they are related. Second, it uses a description of data sources to decide which source can contribute additional knowledge to what is already known about the environment. By querying the sources, the new data obtained is used to grow the current model of the environment. 
