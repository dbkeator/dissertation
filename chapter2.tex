\chapter{What is Context?}

\section{Event Context}
Our justification for the use of context begins with the statement: \textit{For a given user, the correctness of face tags for a photograph containing people she has never met is undefined}. This observation prepares us to understand what context is, and how contextual reasoning assists in tagging photos. The description of any problem domain requires a set of abstract data types, and a model of how these types are related to each other. We \textbf{define} contextual types as those which are semantically different from these data types, but can be directly or indirectly related to them via an extended model which encapsulates the original one. Contextual reasoning assists in the following two ways. \textbf{First}, contextual data restricts the number of people who might appear in the photographs. We can also argue that all the personal data of a user (her profile on Facebook, LinkedIn, email exchanges, phone call logs) provides a reasonable estimate of all these people who might appear in her photos. \textbf{Second}, by reasoning on abstractions in the contextual domain, we can infer conclusions on the original problem. We exploit this property to develop our algorithm in the later sections. Though CueNet can be applied to a variety of recognition problems, we focus on tagging people in personal photos for concreteness, where, the image and person tag form the abstractions in the problem domain. The types used in the contextual domain, but not limited to, are the following:

\begin{itemize}
\item \textbf{Events}: includes description of events like conferences, parties, trips or weddings, and their structure (for example, what kind of sessions, talks and keynotes are occurring within a particular conference).
\item \textbf{Social Relationships}: information about a user's social graph, people whom she corresponds with using email and other messaging services.
\item \textbf{Geographical Proximity}: various tools like Facebook Places, Google Latitude or Foursquare provide information about where people are at a given time.
\end{itemize}

The above classes of contextual data can be obtained from a variety of data sources. Examples of data sources range from mobile phone call logs and email conversations to Facebook messages to a listing of public events at upcoming.com. We classify sources into the following types:

\begin{itemize}
\item \textbf{Personal Data Sources}: include all sources which provide details about the particular user whose photo is to be tagged. Some common examples of personal data sources include Google Calendar, Email and Facebook profile and social graph.
\item \textbf{Social Data Sources}: include all sources which provide contextual information about a user's friends and colleagues. For example, LinkedIn, Facebook and DBLP are some of the commonly used websites with different types of social graphs.
\item \textbf{Public Data Sources}: include all sources which provide information about public organizations (like restaurants, points of interest or football stadiums) or about public events (like fairs, concerts or sports games).
\end{itemize}

Social and public data sources are enormous in size, containing information about billions of events and entities. Trying to use them directly will lead to scalability problems faced by face recognition and verification techniques. But, by using personal data, we can discover which parts of social and public sources are more relevant. For example, if a photo was taken at San Francisco, CA (where the user lives) his family in China is less relevant. Thus, the role of personal information is twofold. \textbf{Firstly}, it provides contextual information regarding the photo. \textbf{Secondly}, it acts as a bridge to connect to social and public data sources to discover interesting people connected to the user who might be present in the event and therefore, the photo.

At this point we will mention the \textbf{temporal relevance} property of a data source. Given a stream of photos taken during a time interval, the source which contributed interesting context for a photo might not be equally useful for the one appearing next. This is because sources tend to focus on a specific set of event types or relationship types, and the two photos might be captured in different events or contains persons with whom the user maintains relations through different sources. For example, two photos taken at a conference might contain a user's friends in the first, but with advisers of these friends in the next. The friends might interact with the user through a social network, but their advisers might not. By using a source like DBLP, the relations between the adviser and friends can be discovered. We say that the temporal relevance of these context sources is \textbf{\textit{low}}. This requirement will play an important role in the design of our framework, as now, sources are not hardwired to photo, but instead need to be discovered gradually.
