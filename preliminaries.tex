% \thesistitle{
%   Pruning Large Search Spaces using Context Networks
% }

\thesistitle {
  Discovering Real-World Context to Tag Personal Photos
}

\degreename{Doctor of Philosophy}

% Use the wording given in the official list of degrees awarded by UCI:
% http://www.rgs.uci.edu/grad/academic/degrees_offered.htm
\degreefield{Computer Science}

% Your name as it appears on official UCI records.
\authorname{Arjun Satish}

% Use the full name of each committee member.
\committeechair{Professor Ramesh Jain}
\othercommitteemembers
{
  Dr. Amarnath Gupta\\
  Professor Nalini Venkatasubramanian\\
  Professor Deva Ramanan\\
  Professor Bill Tomlinson
}

\degreeyear{2013}

\copyrightdeclaration
{
  {\copyright} {\Degreeyear} \Authorname
}

% If you have previously published parts of your manuscript, you must list the
% copyright holders; see Section 3.2 of the UCI Thesis and Dissertation Manual.
% Otherwise, this section may be omitted.
% \prepublishedcopyrightdeclaration
% {
% 	Chapter 4 {\copyright} 2003 Springer-Verlag \\
% 	Portion of Chapter 5 {\copyright} 1999 John Wiley \& Sons, Inc. \\
% 	All other materials {\copyright} {\Degreeyear} \Authorname
% }

% The dedication page is optional.
\dedications
{
  % \textit{And it's whispered that soon, if we all call the tune,} \\
  % \textit{Then the piper will lead us to reason} \\
  % \textit{And a new day will dawn for those who stand long} \\
  % \textit{And the forest will echo with laughter} \\
}

\acknowledgments
{
%  I would like to thank...
%  
%  (You must acknowledge grants and other funding assistance. 
%  
%  You may also acknowledge the contributions of professors and
%  friends.
%  
%  You also need to acknowledge any publishers of your previous
%  work who have given you permission to incorporate that work
%  into your dissertation. See Section 3.2 of the UCI Thesis and
%  Dissertation Manual.)
}


% Some custom commands for your list of publications and software.
\newcommand{\mypubentry}[3]{
  \begin{tabular*}{1\textwidth}{@{\extracolsep{\fill}}p{4.5in}r}
    \textbf{#1} & \textbf{#2} \\ 
    \multicolumn{2}{@{\extracolsep{\fill}}p{.95\textwidth}}{#3}\vspace{6pt} \\
  \end{tabular*}
}
\newcommand{\mysoftentry}[3]{
  \begin{tabular*}{1\textwidth}{@{\extracolsep{\fill}}lr}
    \textbf{#1} & \url{#2} \\
    \multicolumn{2}{@{\extracolsep{\fill}}p{.95\textwidth}}
    {\emph{#3}}\vspace{-6pt} \\
  \end{tabular*}
}

% Include, at minimum, a listing of your degrees and educational
% achievements with dates and the school where the degrees were
% earned. This should include the degree currently being
% attained. Other than that it's mostly up to you what to include here
% and how to format it, below is just an example.
\curriculumvitae
{

\textbf{EDUCATION}
  
  \begin{tabular*}{1\textwidth}{@{\extracolsep{\fill}}lr}
    \textbf{Doctor of Philosophy in Computer Science} & \textbf{2013} \\
    \vspace{6pt}
    University of California, Irvine & \emph{Irvine, CA} \\
    \textbf{MS in Computer Science} & \textbf{2011} \\
    \vspace{6pt}
    University of California, Irvine & \emph{Irvine, CA} \\
    \textbf{BE in Electronics and Communications} & \textbf{2002-2006} \\
    \vspace{6pt}
    Sir MV Institute of Technology & \emph{Bangalore, Karnataka, India} \\
  \end{tabular*}

\vspace{12pt}
\textbf{RESEARCH EXPERIENCE}

  \begin{tabular*}{1\textwidth}{@{\extracolsep{\fill}}lr}
    \textbf{Graduate Research Assistant} & \textbf{2007--2013} \\
    \vspace{6pt}
    University of California, Irvine & \emph{Irvine, California} \\
  \end{tabular*}

\vspace{12pt}
\textbf{TEACHING EXPERIENCE}

  \begin{tabular*}{1\textwidth}{@{\extracolsep{\fill}}lr}
    \textbf{Teaching Assistant} & \textbf{2007--2013} \\
    \vspace{6pt}
    University of California, Irvine & \emph{Irvine, California} \\
  \end{tabular*}

\pagebreak

\textbf{REFEREED CONFERENCE PUBLICATIONS}

  \mypubentry{CueNet: a Context Discovery Framework to Tag Personal Photos}{Mar 2013}{ICMR 2013}  
  \mypubentry{Visualizing Progressive Discovery}{Mar 2013}{ICMR 2013}  
  \mypubentry{Tolkien: An Event Based Storytelling System}{Aug 2009}{VLDB 2009}
  \mypubentry{A New Approach for Adding Browser Functionality}{Jun 2008}{Hypertext 2008}

\vspace{12pt}
\textbf{TECHNICAL REPORTS}

  \mypubentry{Context Networks for Annotating Personal Media}{May 2013}{ESL.UCI.EDU-TR 2013-May/01}
  \mypubentry{Tagging Personal Photos Using Contextual Information}{Aug 2012}{Whitepaper}
  \mypubentry{Lives: A System for Creating Families of Multimedia Stories}{Aug 2010}{MSR-TR-2011-65}
  \mypubentry{Tolkien: Weaving Stories from Personal Media}{Jun 2010}{ESL.UCI.EDU-TR 2010/04/11}

\vspace{12pt}
\textbf{SOFTWARE}

  \mysoftentry{CueNet}{https://github.com/wicknicks/cuenet/}
  {Polyglot implementation of the CueNet ecosystem to tag faces in photos.}

}

% The abstract should not be over 350 words, although that's
% supposedly somewhat of a soft constraint.
\thesisabstract
{
Automatic annotation algorithms assign one or more labels from a candidate search space to a given input object. The collection of these labels is either provided manually, or derived from a set of sources. It is known that the accuracy of annotation algorithms is typically a decreasing function of the number of labels in the search space. In annotating real-world multimedia objects, this search space quickly explodes if not carefully curated. In this work, we present a technique to prune search spaces yet retain a very large number of correct tags for a multimedia object. Specifically, we shall address pruning of search spaces for the problem of tagging faces in the photo media.

Photos capture the momentary state of real-world entities. Any knowledge of the state of entities in the real-world at the time of photos capture can provide very useful context to eliminate candidates in the search space. For example, people in Japan cannot appear in photos taken at Paris, and photos taken at an academic conferences will largely contain people who work in the same field. 

In the real world, relations between entities change rapidly. It is not possible to use a static model of entity relationships to reason over faces in photo instances taken across a wide range of time. Thus, it is imperative that any system first \textit{discover} the set of relationships which are true at the time of photo capture before pruning the search space.

In this dissertation, we adopt a \textbf{relation-centric view} of real-world contextual information to design and implement a framework, \textbf{CueNet} to discover real-world context. The primary contribution of this dissertation is a \textbf{progressive discovery algorithm} to construct representations of dynamic real-world entity relationships. We refer to such representations as \textbf{context networks}. We also present a \textbf{method to rank} entities in the case of incomplete contextual information. We shall extend the concepts behind the PageRank algorithm, to rank webpages, to rank real-world entities, and design score propagation techniques based on event semantics. 

We present \textbf{experiments} on both real-world photos and simulated events. We implement a system using CueNet to tag thousands of personal photos captured by multiple users, to study the efficacy of using context discovery in pruning search spaces. We also present simulation experiments to measure the time and space complexity of large scale discovery operations. We present the convergence characteristics and results on our real-world datasets to show its utility. 

Finally, we outline the known issues faced by our platform and present a set of new applications can be designed at the basis of context discovery techniques.
}


%%% Local Variables: ***
%%% mode: latex ***
%%% TeX-master: "thesis.tex" ***
%%% End: ***
