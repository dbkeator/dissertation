% \thesistitle{
%   Pruning Large Search Spaces using Context Networks
% }

\thesistitle {
  Discovering Real-World Context to Tag Personal Photos
}

\degreename{Doctor of Philosophy}

% Use the wording given in the official list of degrees awarded by UCI:
% http://www.rgs.uci.edu/grad/academic/degrees_offered.htm
\degreefield{Computer Science}

% Your name as it appears on official UCI records.
\authorname{Arjun Satish}

% Use the full name of each committee member.
\committeechair{Professor Ramesh Jain}
\othercommitteemembers
{
  Dr. Amarnath Gupta\\
  Professor Nalini Venkatasubramanian\\
  Professor Deva Ramanan\\
  Professor Bill Tomlinson
}

\degreeyear{2013}

\copyrightdeclaration
{
  {\copyright} {\Degreeyear} \Authorname
}

% If you have previously published parts of your manuscript, you must list the
% copyright holders; see Section 3.2 of the UCI Thesis and Dissertation Manual.
% Otherwise, this section may be omitted.
% \prepublishedcopyrightdeclaration
% {
% 	Chapter 4 {\copyright} 2003 Springer-Verlag \\
% 	Portion of Chapter 5 {\copyright} 1999 John Wiley \& Sons, Inc. \\
% 	All other materials {\copyright} {\Degreeyear} \Authorname
% }

% The dedication page is optional.
\dedications
{
  % \textit{And it's whispered that soon, if we all call the tune,} \\
  % \textit{Then the piper will lead us to reason} \\
  % \textit{And a new day will dawn for those who stand long} \\
  % \textit{And the forest will echo with laughter} \\
}

\acknowledgments
{
%  I would like to thank...
%  
%  (You must acknowledge grants and other funding assistance. 
%  
%  You may also acknowledge the contributions of professors and
%  friends.
%  
%  You also need to acknowledge any publishers of your previous
%  work who have given you permission to incorporate that work
%  into your dissertation. See Section 3.2 of the UCI Thesis and
%  Dissertation Manual.)
}


% Some custom commands for your list of publications and software.
\newcommand{\mypubentry}[3]{
  \begin{tabular*}{1\textwidth}{@{\extracolsep{\fill}}p{4.5in}r}
    \textbf{#1} & \textbf{#2} \\ 
    \multicolumn{2}{@{\extracolsep{\fill}}p{.95\textwidth}}{#3}\vspace{6pt} \\
  \end{tabular*}
}
\newcommand{\mysoftentry}[3]{
  \begin{tabular*}{1\textwidth}{@{\extracolsep{\fill}}lr}
    \textbf{#1} & \url{#2} \\
    \multicolumn{2}{@{\extracolsep{\fill}}p{.95\textwidth}}
    {\emph{#3}}\vspace{-6pt} \\
  \end{tabular*}
}

% Include, at minimum, a listing of your degrees and educational
% achievements with dates and the school where the degrees were
% earned. This should include the degree currently being
% attained. Other than that it's mostly up to you what to include here
% and how to format it, below is just an example.
\curriculumvitae
{

\textbf{EDUCATION}
  
  \begin{tabular*}{1\textwidth}{@{\extracolsep{\fill}}lr}
    \textbf{Doctor of Philosophy in Computer Science} & \textbf{2013} \\
    \vspace{6pt}
    University of California, Irvine & \emph{Irvine, CA} \\
    \textbf{MS in Computer Science} & \textbf{2011} \\
    \vspace{6pt}
    University of California, Irvine & \emph{Irvine, CA} \\
    \textbf{BE in Electronics and Communications} & \textbf{2002-2006} \\
    \vspace{6pt}
    Sir MV Institute of Technology & \emph{Bangalore, Karnataka, India} \\
  \end{tabular*}

\vspace{12pt}
\textbf{RESEARCH EXPERIENCE}

  \begin{tabular*}{1\textwidth}{@{\extracolsep{\fill}}lr}
    \textbf{Graduate Research Assistant} & \textbf{2007--2013} \\
    \vspace{6pt}
    University of California, Irvine & \emph{Irvine, California} \\
  \end{tabular*}

\vspace{12pt}
\textbf{TEACHING EXPERIENCE}

  \begin{tabular*}{1\textwidth}{@{\extracolsep{\fill}}lr}
    \textbf{Teaching Assistant} & \textbf{2007--2013} \\
    \vspace{6pt}
    University of California, Irvine & \emph{Irvine, California} \\
  \end{tabular*}

\pagebreak

\textbf{REFEREED CONFERENCE PUBLICATIONS}

  \mypubentry{CueNet: a Context Discovery Framework to Tag Personal Photos}{Mar 2013}{ICMR 2013}  
  \mypubentry{Visualizing Progressive Discovery}{Mar 2013}{ICMR 2013}  
  \mypubentry{Tolkien: An Event Based Storytelling System}{Aug 2009}{VLDB 2009}
  \mypubentry{A New Approach for Adding Browser Functionality}{Jun 2008}{Hypertext 2008}

\vspace{12pt}
\textbf{TECHNICAL REPORTS}

  \mypubentry{Context Networks for Annotating Personal Media}{May 2013}{ESL.UCI.EDU-TR 2013-May/01}
  \mypubentry{Tagging Personal Photos Using Contextual Information}{Aug 2012}{Whitepaper}
  \mypubentry{Lives: A System for Creating Families of Multimedia Stories}{Aug 2010}{MSR-TR-2011-65}
  \mypubentry{Tolkien: Weaving Stories from Personal Media}{Jun 2010}{ESL.UCI.EDU-TR 2010/04/11}

\vspace{12pt}
\textbf{SOFTWARE}

  \mysoftentry{CueNet}{https://github.com/wicknicks/cuenet/}
  {Polyglot implementation of the CueNet ecosystem to tag faces in photos.}

}

% The abstract should not be over 350 words, although that's
% supposedly somewhat of a soft constraint.
\thesisabstract
{
The search spaces of real world AI problems are extremely large. The strategy adopted to prune large search spaces is to accurately model the environment, and reason which parts of the search space can be pruned without hurting the performance of the decision making algorithm. Although relatively easier in virtual environments, modeling dynamic real world environments is a very challenging problem.

Consider the example of tagging faces in a person's photo album. The search space contains a few billion potential candidates. Any algorithm which attempts to directly tag one of billion people in a given photo will perform poorly. But, given a complete model of the world, the search space needs to be limited only to the entities who were present close to the camera's field of view at the time of photo capture. Now, the algorithm needs to decide over few tens of candidates as opposed to billions.

In this dissertation, we present \textit{Context Networks}, a representation of real world environments used to prune search spaces for real world AI problems, and a novel \textit{Progressive Discovery} algorithm to construct such networks from heterogenous data sources. We facilitate our following discussions through an example system to tag faces in personal photos. We discuss the architecture of a complete system to model data sources, construct context networks for a given AI problem (which in the case of face tagging would be a personal photo and the exhaustive search space) and provides a pruned version of the search space most relevant to the problem. We also present experiments to quantitatively demonstrate the efficacy of our algorithm.
}


%%% Local Variables: ***
%%% mode: latex ***
%%% TeX-master: "thesis.tex" ***
%%% End: ***
